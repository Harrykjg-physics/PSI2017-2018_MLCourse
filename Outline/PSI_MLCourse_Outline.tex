\documentclass[letterpaper]{scrartcl}	
\usepackage[top=0.8in, bottom=0.9in, left=0.9in, right=0.9in]{geometry}
\pagestyle{empty}

%--------------------------------------------------------------
% We need this package, part of the KOMA class, for the custom
% headings.
%--------------------------------------------------------------
%\usepackage{scrpage2}	
		

%--------------------------------------------------------------
% One of many packages you can use if you want to include
% graphics.
%--------------------------------------------------------------
\usepackage{graphicx}			

%--------------------------------------------------------------
% The AMS packages offer a number of nice fonts, 
% environments for formatting multiline equations, etc.
%--------------------------------------------------------------
\usepackage{amsmath}			
\usepackage{amsfonts}
\usepackage{amssymb}
\usepackage{amsthm}


%--------------------------------------------------------------
% This package is used to define custom colours.
%--------------------------------------------------------------
\usepackage{xcolor}

%--------------------------------------------------------------
% This is where we define the custom title. The image that is
% placed on the left-hand-side of the title, PILogo.pdf in
% this case, should be in the same directory as this file. Note
% that you can always use hyperlinks for the Title, Semester,
% and Author fields, below, in case you want to link to a seminar
% web page or a lecturer's email address.
%--------------------------------------------------------------
\titlehead{%
	\begin{minipage}[b]{3.0cm}
	\includegraphics*[height=1.3cm]{PSIletterhead.eps}% %\includegraphics*[height=2.8cm]{PI_logo_2.png}%
	\end{minipage}
	\hfill
	\begin{minipage}[b]{11cm}
	\vspace*{-2cm}
	\begin{flushright}
		\usekomafont{descriptionlabel}
		\Large Machine Learning for Many-Body Physics \\
		\large April 9--27, 2018\\
		Course Outline
	\end{flushright}
	\end{minipage}
	\\[-3mm]
	%\hrule
	\vspace{-3mm}
}
% -----------
\usepackage{enumitem}

\begin{document}

%\scalefont{1.35}

\title{}
\date{}
\maketitle
\thispagestyle{empty}

%%%%%%%%%%%%%%%%%%%%%%%%%%%%%%%%%%%%%%%%%%%%%%%%
%%%%%%%%%%%%%%%%%%%%% OBJECTIVE %%%%%%%%%%%%%%%%%%%%
%%%%%%%%%%%%%%%%%%%%%%%%%%%%%%%%%%%%%%%%%%%%%%%%
\vspace*{-3cm}
\noindent\usekomafont{descriptionlabel}{\large Objective:} 
\normalfont This course is designed to introduce modern machine learning techniques for studying classical and quantum many-body problems encountered in condensed matter, quantum information, and related fields of physics. 
Lectures will emphasize relationships between statistical physics and machine learning. 
Tutorials and homework assignments will focus on developing programming skills for machine learning using Python and TensorFlow.

%%%%%%%%%%%%%%%%%%%%%%%%%%%%%%%%%%%%%%%%%%%%%%%%
%%%%%%%%%%%%%%%%%%% TEACHING TEAM %%%%%%%%%%%%%%%%%%%
%%%%%%%%%%%%%%%%%%%%%%%%%%%%%%%%%%%%%%%%%%%%%%%%
\vspace*{0.8cm}
\noindent\usekomafont{descriptionlabel}{\large Lecturers:} \normalfont 
Lauren Hayward Sierens, Juan Carrasquilla, Roger Melko, Giacomo Torlai \\
\noindent\usekomafont{descriptionlabel}{\large PSI Fellow:} \normalfont Lauren Hayward Sierens

%%%%%%%%%%%%%%%%%%%%%%%%%%%%%%%%%%%%%%%%%%%%%%%%
%%%%%%%%%%%%%%%%%%% OUTLINE OF TOPICS %%%%%%%%%%%%%%%%%%%
%%%%%%%%%%%%%%%%%%%%%%%%%%%%%%%%%%%%%%%%%%%%%%%%
\vspace*{0.8cm}
\noindent\usekomafont{descriptionlabel}{\large Outline of topics:} 
\normalfont 
\begin{itemize} 
\setlength\itemsep{0em}
\item Lattice models for statistical physics 
\item Monte Carlo methods 
\item Supervised and unsupervised learning 
\item Feedforward and convolutional neural networks
\item Data visualization and clustering
\item Generative modelling and Boltzmann machines
\item Quantum state tomography using neural networks
\end{itemize}

%%%%%%%%%%%%%%%%%%%%%%%%%%%%%%%%%%%%%%%%%%%%%%%%
%%%%%%%%%%%%%%%%%%%% COURSE WIKI %%%%%%%%%%%%%%%%%%%%
%%%%%%%%%%%%%%%%%%%%%%%%%%%%%%%%%%%%%%%%%%%%%%%%
\vspace*{0.6cm}
\noindent\usekomafont{descriptionlabel}{\large Course Wiki:} 
\normalfont Course materials (including tutorials, homework assignments, references and links to lectures) 
will be posted on the PSI wiki 
(https://perimeterinstitute.ca/psi{\textunderscore}portal).

%%%%%%%%%%%%%%%%%%%%%%%%%%%%%%%%%%%%%%%%%%%%%%%%
%%%%%%%%%%%% COURSE REQUIREMENTS AND ASSESSMENT %%%%%%%%%%%
%%%%%%%%%%%%%%%%%%%%%%%%%%%%%%%%%%%%%%%%%%%%%%%%
\vspace*{0.8cm}
\noindent\usekomafont{descriptionlabel}{\large Course Requirements and Assessment:} 
\normalfont The final grade for this course will be either Credit or No Credit. 
In order to receive credit for this course, each student must
\begin{itemize}
\setlength{\itemsep}{0.5ex}
\item participate in all six \usekomafont{descriptionlabel}{tutorials}. \normalfont Students are expected to attend tutorials and discuss tutorial problems with other students. 
In cases where a student needs to miss a tutorial, arrangements must be made with Lauren before the tutorial.

\item submit and pass all \usekomafont{descriptionlabel}{homework assignments}. \normalfont Students are expected to submit each assignment by the posted deadline using the online submission system. 
In cases where a student needs additional time to finish an assignment, arrangements must be made with Lauren before the deadline.
\end{itemize}

%%%%%%%%%%%%%%%%%%%%%%%%%%%%%%%%%%%%%%%%%%%%%%%%
%%%%%%%%%%%%%%%%%% ACADEMIC INTEGRITY %%%%%%%%%%%%%%%%%
%%%%%%%%%%%%%%%%%%%%%%%%%%%%%%%%%%%%%%%%%%%%%%%%
\vspace*{0.6cm}
\noindent\usekomafont{descriptionlabel}{\large Academic Integrity:} \normalfont All students are expected to know, understand, and follow the academic integrity policies detailed on the University of Waterloo Academic Integrity website \\
(https://uwaterloo.ca/academic-integrity/).

\end{document}
